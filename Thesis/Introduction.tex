%% Einleitung.tex
%%
\chapter{Introduction}
\label{ch:Introduction}
\section{Motivation}
%% ============================
Software and hardware control of the traditional network cannot be managed independently as the functionality of its basic building blocks, for instance, switches, hubs and routers are implemented with dedicated network appliance and hardware. These devices evolve slowly as most of their hardware and software components are closed, proprietary boxes which cannot be centrally managed and have to be configured individually. Provisioning and roll-out mechanism of traditional network devices like routers and switches are time consuming and also prone to errors\cite {OpenFlow}.
To address these issues, networking researchers have developed an open standard platform known as OpenFlow\cite{OpenFlow}.  OpenFlow has provided a standard interface to install and configure rules in OpenFlow based network switches. These rules are installed in switches using programmable centralized controller.

Software-Defined Networking (SDN) is an emerging technology that is dynamic, software-centric, centrally manageable, cost-effective and highly adaptable making it an ideal alternate to traditional networking\cite{OpenFlow}. A SDN architecture decouples the network control plane from the forwarding data plane and uses OpenFlow protocol for communicating between controller and switches. Due to the advantages of SDN over legacy networks, OpenFlow based SDN solutions are being deployed in data center, enterprise and campus networks. However, there are limitations to SDN based real networks such as scalability and reliability which are also addressed in my thesis.

%% ==============================
\section{Objective}
%% ==============================
\label{ch:Introduction:sec:Objective}

In this thesis, I am deploying and evaluating SDN based solution in real networks. My task is to migrate selected functionality i.e., switching, routing, firewall of an exemplary traditional network into SDN with adequate interfaces. Due to its recent deployment in production networks SDN concepts still need maturation. Best practice for tackling various limitations and problems in SDN domain are to be addressed in an efficient way. For instance, implementing network policies that deals with access control and routing policy that deals with traffic paths are not solved in a satisfying way. The "One Big Abstraction" is a concept that allows defining efficient rules for the network in a centralized way\cite{TB98}. SDN solution will also include this "One Big Abstraction" concept. The "One Big Abstraction" is a concept that allows defining efficient rules for the network in a centralized way\cite{TB98}.SDN solution will also include  this "One Big Abstraction" concept.
The Motivation of this thesis is to take advantage of the centralized approach of SDN to develop network services like switching, routing, firewall, traffic engineering and load balancing algorithm independent of propriety based protocols. The SDN network will be deployed in our campus with effective network management techniques and efficient network policies which will resolve the shortcoming of the already deployed traditional network such as scalability and network management. The objective of the thesis is to implement vital network services like switching, routing,firewall, traffic engineering and load balancing for SDN based network. The SDN network will be deployed in our campus with effective network management techniques and efficient network policies which will resolve the shortcoming of the already deployed traditional network such as scalability and network management. 
%% ==============================
\section{Thesis Structure}
%% ==============================
\label{ch:Introduction:sec:Thesis Structure}

The work in this thesis is divided into three phases. First part of this thesis is comprised of chapters 2 and 3 in which I have developed and tested different applications for use cases such as routing, switching and firewall on Mininet software. Mininet is open source software that allows emulating an entire network. It is the main tool which I have used for the testbed environments to design, undergo and verify different use cases. Second phase consisting of chapter 4 and 5 involves migration of the campus network topology into the testbed environment which also includes testing of developed services on the campus topology. Chapter 6 and 7 of the final phase will consist of interpretation of results and future work in this regard 

\textit{ {\color{red}
\begin{itemize}
\item Description of chapter More?
\item Appendix? (justification of selection)
\end{itemize}
}}
