%% analyse.tex
%%
\chapter{Background}
\label{ch:Analyse}
\section{Introduction}
%% ==============================
This chapter features theoretical aspect about the technologies and tools that are used to reach the goal. In the following section I introduce the concept of Software Defined Networking (SDN) and OpenFlow protocol. In section 2.3 publicly available open source SDN controllers are discussed, compared and one of the controllers is selected.    
%% ==============================
\section{Software Defined Networking (SDN)}
%% ==============================
\label{ch:Analyse:sec:Abschnitt1}

Software Defined Networking (SDN) is a concept that involves separation of control plane from the data plane. SDN is an innovating technology that can be used to solve scalability and management issues involved in the legacy networks\cite{OpenFlow}.  In SDN, the control plane is the logic that controls the forwarding behavior in the network. Control plane can be seen as the network brain as it is responsible for making decisions involving the path and the way the network traffic is sent. Control plane is also involved in populating switching flow tables according to defined instructions in the application and which can also help in achieving tasks of load balancing and firewall configuration in the network. Data plane on the other hand forwards the traffic according to the logic of the control plane. Examples of the data plane tasks are Internet Protocol (IP) forwarding and switching\cite {OpenFlow}.

The key feature of SDN architecture, decoupling of control from data plane, leads to many advantages over legacy networks. The control plane is directly programmable as it is decoupled from the forwarding functions of the data plane. SDN networks are more agile as the network administrator is able to dynamically manipulate traffic flow as per demand and requirement. Separation of data and control plane also help to centrally manage the network as the intelligence control is logically centralized in software based SDN controllers. A global view of the network is maintained by the SDN based controllers, which appears to applications and policy engines as a single, logical switch. Another key advantage of SDN network is that each of the control or data plane can evolve independently and are not vendor specific. The software control of the network can be managed independently of the hardware\cite {OpenFlow}.All of these advantages leads to promising SDN benefits that are applied in proper and practical way in this thesis.  

A basic SDN architecture consists of three abstraction layers: application, control and infrastructure layer. The SDN architecture differs from legacy solutions by building networks from these three layers. Figure 2.1 is a graphical represent of such a SDN architecture

\begin{figure}
	\centering
	\includegraphics*[scale=0.30] {sdn.png}
    \caption{Graphical Representation of the SDN Architecture} 
\end{figure}

The infrastructure layer acts as the foundation for SDN architecture. It consists of physical network devices such as switches and virtual elements like routers running on the same physical switch. These network devices are responsible to forward traffic according to the rules installed in their flow table by the controller. These devices use the OpenFlow protocol as a method to implement the traffic forwarding rules.\cite{Hp}. Control layer is present above the infrastructure layer without which the network devices cannot receive instruction. The control plane is separated from the underlying infrastructure to provide a single centralized view of the entire network. The control layer communicates with the infrastructure layer via OpenFlow interface. In addition to this, control layer consist of a centralized controller that provide different type of network functionalities such as access list for firewall protection and topology discovery. At top of SDN architecture stack there is an application layer consisting of user defined business applications that interact with control layer.

The application layer consists of north bound application programmable interface (API)  that enables communication between the control layer and application layer. Northbound APIs are not standardized at the moment and present a network abstraction interface to the application and management system\cite{Hp}. Northbound APIs can be programmed to enable basic network functions and services like loop avoidance, path selection, switching and routing. They can also be used to develop orchestration tools to manage network services in the cloud. On the contrary, south bound APIs are used for interacting between the control and infrastructure layer. One example of south bound API is OpenFlow interface that is used by the controller to define and instruct the switches and network devices at the infrastructure layer.

OpenFlow is one of the key technologies to realize SDN. It provides an open interface between control and data plane. In OpenFlow based SDN architecture both controller and switches need to support OpenFlow protocol through which they can communicate with each other. Controller communicates with switches in a secure channel and the OpenFlow protocol effectively instructs the switches to update the flow table entries according to controller logic.  Flow table is updated and the switch forwards the traffic according to the flow table entries installed by the controller. OpenFlow enables a SDN external controller to insert forwarding rules in a switch’s action tables, rather than having switch learn its own rules.


\textit{ {\color{red}
\begin{itemize}
\item Use of Open Flow in general
\item How I will be using (OpenFlow)
\item OpenFlow operation in essence its working.
\end{itemize}
}}

%% ==============================
\section{SDN  Controllers}
%% ==============================
\label{ch:Analyse:sec:Abschnitt2}

OpenFlow based SDN controllers communicate with switches through OpenFlow protocol. All the logic and smartness of the behaviour of the network is controlled at the controller. There are many different type SDN controllers in the market some of them are NOX/POX, Floodlight. Open Daylight, Pyretic, Frenetic, Procera, RouteFlow and Trema. Before selecting the type of SDN controller there are many consideration that is required for an appropriate controller. One of them is programming language which in some cases can also affect the performance\cite{Hp}.Second is the stability of the code to avoid system crash more often. In case of Open source SDN controller they should have very active base community support. Finally, the foremost important aspect involves the architecture of an SDN controller that is based on support of southbound API, northbound API/the policy layer , support for OpenStack or multitenant cloud environment and whether the controller is intended to be used as operating system or was designed for specific purpose like education, research or production.

\subsection { Types of SDN Controllers}

I will discuss some of the controllers. NOX is the first generation openflow controller, key features include support for opensource based and widely used. POX is python implementation of NOX. Advantage of NOX is widely used, maintained and supported and is relatively easily to use and write. Disadvantage of POX is performance. POX is mainly use in research, experimentation and documentation.
Flood light is an OpenSource java controller that supports openflow version 1.0 which is maintained by BigSwitch. It is a production level controller and provides support to openstack. Another SDN controller that is being actively used is OpenDay light. Keys feature include support common industry platform and is robust, extensible open source database with common abstraction for northbound capabilities. Advantages of OpenDay light is  Industry acceptance, integration with openstack , cloud application. Ryu supports all version of openflow and aims to be Operating System \ldota

\textit{ {\color{red}
\begin{itemize}
\item Different type of Controllers
\item Description about them where they are used and how
\item Comparison in the table
\end{itemize}
}}



\subsection{Ryu SDN Controller}


\textit{ {\color{red}
\begin{itemize}
\item what is Ryu?
\item Why Ru? (justification of selection)
\item How it works? (brief explanation about the structure, Architecture and it's behaviour)
\item main parts involved with the algorithm (listeners of events, forwarding module...)
\end{itemize}
}}

%% Ryu is an OpenSource SDN framework. Ryu is a platform to build  SDN applications which can be used to manage the network. Ryu controller consists of well-defined northbound APIs and useful libraries. It becomes easier for the developer to develop and implement  own SDN applications on top of Ryu.  Ryu works as an operating system as it builds application on top of the ryu to control our network resource. Ryu supports open flow  and its source code is written on python. Ryu development is open. Ryu is to become defact sdn framework , and provide cloud computing . open stack is an example. Ryu is designed and implemented in production use for large production environment using code quality, functionality. 
%% Features of Ryu is generality as it is vendor-neutral, supports open interface ( e.g OpenFlow) support by many switch vendors.It is agile as framework for an application development and is a component base controller. Ryu architecture can be seen as below

%%Architecure 

%%It can be seen. It not only support openflow but also other. We can control existing devices by RYU as it supports other protocols. Ryu Architecture can be seen is supported libraries with internal. Network operator can get switch state and flows through rest API and openstack provide sources for rest api. two development style for ryu when you implement.It has some built in APIs which we can use for our applications example topology discover , routing and firewall.for example switch state.two development style , business application at top of ryu to define own api. it communicate with user defined api. we can use external or internal api.

\subsection{Introduction to Ryu Modules}
